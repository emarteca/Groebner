
\documentclass[letterpaper,12pt,titlepage,oneside,final]{book}
 

\newcommand{\package}[1]{\textbf{#1}} 
\newcommand{\cmmd}[1]{\textbackslash\texttt{#1}} 
\newcommand{\href}[1]{#1} 


\usepackage{ifthen}
\usepackage{indentfirst}
\newboolean{PrintVersion}
\setboolean{PrintVersion}{false} 

\usepackage{amsmath,amssymb,amstext}
\usepackage[pdftex]{graphicx}

\usepackage{lipsum}
\newcounter{examplecounter}
\newenvironment{example}{\begin{quote}%
    \refstepcounter{examplecounter}%
  \textbf{Example \arabic{examplecounter}}%
  \quad
}{%
\end{quote}%
}

\usepackage[pdftex,letterpaper=true,pagebackref=false]{hyperref} 
		
\hypersetup{
    plainpages=false,       % needed if Roman numbers in frontpages
    pdfpagelabels=true,     % adds page number as label in Acrobat's page count
    bookmarks=true,         % show bookmarks bar?
    unicode=false,          % non-Latin characters in Acrobat’s bookmarks
    pdftoolbar=true,        % show Acrobat’s toolbar?
    pdfmenubar=true,        % show Acrobat’s menu?
    pdffitwindow=false,     % window fit to page when opened
    pdfstartview={FitH},    % fits the width of the page to the window
    pdftitle={uWaterloo\ LaTeX\ Thesis\ Template},    % title: CHANGE THIS TEXT!
%    pdfauthor={Author},    % author: CHANGE THIS TEXT! and uncomment this line
%    pdfsubject={Subject},  % subject: CHANGE THIS TEXT! and uncomment this line
%    pdfkeywords={keyword1} {key2} {key3}, % list of keywords, and uncomment this line if desired
    pdfnewwindow=true,      % links in new window
    colorlinks=true,        % false: boxed links; true: colored links
    linkcolor=blue,         % color of internal links
    citecolor=green,        % color of links to bibliography
    filecolor=magenta,      % color of file links
    urlcolor=cyan           % color of external links
}
\ifthenelse{\boolean{PrintVersion}}{   
\hypersetup{	% override some previously defined hyperref options
%    colorlinks,%
    citecolor=black,%
    filecolor=black,%
    linkcolor=black,%
    urlcolor=black}
}{} % end of ifthenelse (no else)


\setlength{\marginparwidth}{0pt} 
\setlength{\marginparsep}{0pt}
\setlength{\evensidemargin}{0.125in}
\setlength{\oddsidemargin}{0.125in}
\setlength{\textwidth}{6.375in} 
\raggedbottom


\setlength{\parskip}{\medskipamount}


\renewcommand{\baselinestretch}{1} 

\let\origdoublepage\cleardoublepage
\newcommand{\clearemptydoublepage}{%
  \clearpage{\pagestyle{empty}\origdoublepage}}
\let\cleardoublepage\clearemptydoublepage

%======================================================================
%   L O G I C A L    D O C U M E N T -- the content of your thesis
%======================================================================
\begin{document}


%======================================================================
\chapter{Introduction}
%======================================================================

%----------------------------------------------------------------------
\section{Brief Overview and History of Groebner Bases}
%----------------------------------------------------------------------
 
pls no

\section{Intermediate Expression Swell}

Expression swell is a problem emergent in computer algebra.

Working with exact computations, where the numerical precision used is not bounded by the language, provides a platform for situations where as computations proceed the values stored keep increasing in size.  Consider multiplication of integers as an example.  Multiplying two n-digit integers results in a 2n-digit integer.  In computations involving sequential multiplication of integers, the product would get exponentially large.  Another example of expression swell can be seen in rational addition. Consider the following demonstrative example:

\begin{equation*}
  \frac{100}{101} + \frac{101}{102} = \frac{20401}{10302}
\end{equation*}

Here, the initial numerators and denominators are all 3-digit numbers, however their sum is a rational number with the numerator and denominator both 5 digits.  Addition of rational numbers results in expression swell as the greatest common denominator increases as more values are included in the sum.

This is not a problem in classic numerical programming languages where the precision of values stored is bounded by the language specifications.  However, when using a symbolic language where computations are exact, the precision is arbitrary and depends directly on the values stored.  Because of this essentially limitless precision, the results are accurate, however expression swell could potentially result in a large amount of memory being required to store values.

Intermediate expression swell is the special case of expression swell where the intermediate results of a computation have increasingly large values (i.e. suffer from expression swell), however this is not reflected in the final result.  This is not as simple to detect as regular expression swell; to be sure some or all of the intermediate computations must be displayed. 

Groebner basis computations are known to suffer from intermediate expression swell.  Generally the output basis does not reflect the size of the coefficients of the intermediate polynomials.  The intermediate coefficient growth is related to many factors of the original basis: degree, number of polynomials, initial coefficient size.  The values can grow to impractical size, slowing down the execution of the computation and perhaps even halting it if the program runs out of memory for storage.  

Consider the following demonstrative example, as provided by Arnolds:

\begin{eqnarray*}
  f_1 &=& 8x^2y^2 + 5xy^3 + 3x^3z + x^2yz\\
  f_2 &=& x^5 + 2y^3z^2 + 13y^2z^3 + 5yz^4\\ 
  f_3 &=& 8x^3 + 12y^3 + xz^2 + 3\\
  f_4 &=& 7x^2y^4 + 18xy^3z^2 + y^3z^3
\end{eqnarray*} 

As the input basis, while the output basis is computed as:

\begin{eqnarray*}
  g_1 &=& x\\
  g_2 &=& y^3 + \frac{1}{4}\\ 
  g_3 &=& z^2\\
\end{eqnarray*} 

The coefficients and for these polynomials are all less than 10; the resulting basis for ${B = [f_1, f_2, f_3, f_4]}$, with degree based monomial ordering\footnote{Referred to in her paper as DegRevLex, degree reverse lexicographical ordering; however this is equivalent to degree based ordering tdeg in Maple} x > y > z, given by ${G = [g_1, g_2, g_3]}$ also has small degree and coefficients.  However, the intermediate computations contain polynomials with very large coefficients, rational numbers commonly of 80 000 digits in both the numerator and denominator (CITE ARNOLDS HERE).  

This is a common occurence in Groebner basis computations.  The size of the intermediate coefficients is not generally reflected in the final output of the normalized basis; it also depends on many factors (degree, initial coefficient size, initial basis size, initial ideal, monomial ordering chosen) and as such is hard to avoid.  Instead of attempting the difficult task of working around it, another solution to this problem will be attempted.



\chapter{Background}

There is some background mathematics necessary to the computation and discussion of Groebner bases.  This will build off assumed knowledge of basic groups and rings in order to explain the ideal theory needed.

\section{Polynomials and Polynomial Ideals}

\section{Monomial Ordering}

Groebner bases are computed with respect to a particular monomial ordering; such an ordering must be defined for a given polynomial space.  For univariate polynomials, the ordering is fairly intuitive.

For polynomials ${p(x), q(x) \in}$ polynomial space ${D[x], \, p(x) = a_n x^n + a_{n-1} x^{n-1} + \ldots + a_1 x + a_0, \, a_i \in D \\ \, q(x) = b_m x^m + b_{m-1} x^{m-1} + \ldots + b_1 x + b_0, \, b_i \in D}$, the ordering is defined as 
\begin{equation*}
  p(x) > q(x) \iff \, \textrm{either} \, n > m \, \textrm{or} \, n = m \; \textrm{and} \; \exists \, a_i > b_i \neq 0, \, \textrm{where} \, a_k = b_k \, \forall k > i 
\end{equation*}
This is a rather complex notation for the intuitive idea for univariate polynomial ordering.  In simpler terms, polynomial ${p > }$ polynomial ${q}$ if ${p}$ has a higher degree, or if they have the same degree, then if ${p}$ has higher values for one or more of its higher order coefficients.

\begin{example}\label{ex: Univariate ordering}
  Consider basic polynomials over the integers.
  \begin{equation*}
    2x^2 + 3x > 5x
  \end{equation*}
  Since the first polynomial is of higher degree.\\
  \begin{equation*}
    3x^2 + 2x > 2x^2 + 5x + 1
  \end{equation*}
  Since the polynomials are of equal degree, but ${3 > 2}$\\
  \begin{equation*}
    3x^2 + 2x > 3x^2 + 5
  \end{equation*}
  Since the polynomials are of equal degree but ${2x}$ is of higher degree than ${5}$
\end{example}

The idea of ordering can be extended to multivariate polynomials, however this requires slightly more work, as the concept is not as intuitive.  While ordering for univariate polynomials is strictly degree-based, when considering multivariate polynomials, the ordering must be considered both degree-wise and lexicographically.  The same polynomials can be ordered in various different ways depending on the monomial ordering chosen; this in turn can have large effects on the efficiency of algorithms working over this polynomial space, including the Groebner basis algorithms.

Whether degree-based or lexicographical (known henceforth as lex) ordering is selected, the variables must still be ordered with respect to each other i.e. an ordering must be chosen.  Then, in degree ordering the total degree is the first factor considered; so, monomials with a higher total degree are considered greater 

To explain the differences and functionality of degree and lex ordering, demonstrative examples will be used.  

\begin{example}\label{ex: Multivariate degree ordering}
  Consider the polynomials ${x^2y^2, xy^3, x^3y, x^4y}$, with ordering ${x > y}$.\\
  The degree-based ordering would be as follows:
  ${x^4y >}$ everything else, since it has total degree 5 (sum of exponents) while the rest of the monomials have total degree 4.\\
  Then, ${x^3y > x^2y^2 > xy^3}$ since, when the total degrees are identical, then the ordering is based on the degree of the "greater" variable.

  Similarly, if the initial variable ordering chosen was ${y > x}$, the ordering would be as follows:
  ${x^4y >}$ everything else, since total degree is still 5.  Then, ${xy^3 > x^2y^2 > x^3y}$, since now ordering is with respect to y i.e. y is the "greater" variable.
\end{example}



\section{Groebner Basis Characterization}





%----------------------------------------------------------------------
% END MATERIAL
%----------------------------------------------------------------------

% B I B L I O G R A P H Y
% -----------------------

% The following statement selects the style to use for references.  It controls the sort order of the entries in the bibliography and also the formatting for the in-text labels.
\bibliographystyle{plain}
% This specifies the location of the file containing the bibliographic information.  
% It assumes you're using BibTeX (if not, why not?).
\cleardoublepage % This is needed if the book class is used, to place the anchor in the correct page,
                 % because the bibliography will start on its own page.
                 % Use \clearpage instead if the document class uses the "oneside" argument
\phantomsection  % With hyperref package, enables hyperlinking from the table of contents to bibliography             
% The following statement causes the title "References" to be used for the bibliography section:
\renewcommand*{\bibname}{References}

% Add the References to the Table of Contents
\addcontentsline{toc}{chapter}{\textbf{References}}

\bibliography{uw-ethesis}
% Tip 5: You can create multiple .bib files to organize your references. 
% Just list them all in the \bibliogaphy command, separated by commas (no spaces).

% The following statement causes the specified references to be added to the bibliography% even if they were not 
% cited in the text. The asterisk is a wildcard that causes all entries in the bibliographic database to be included (optional).
\nocite{*}

\end{document}